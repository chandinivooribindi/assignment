\let\negmedspace\undefined
\let\negthickspace\undefined
\documentclass{article}
\usepackage{cite}
\usepackage{amsmath,amssymb,amsfonts,amsthm}
\usepackage{algorithmic}
\usepackage{graphicx}
\usepackage{textcomp}
\usepackage{xcolor}
\usepackage{txfonts}
\usepackage{float}
\usepackage{listings}
\usepackage{enumitem}
\usepackage{mathtools}
\usepackage{gensymb}
\usepackage{tfrupee}
\usepackage[breaklinks=true]{hyperref}
\usepackage{tkz-euclide} % loads  TikZ and tkz-base
\usepackage{listings}
\usepackage{gvv}
%
%\usepackage{setspace}
%\usepackage{gensymb}
%\doublespacing
%\singlespacing

%\usepackage{graphicx}
%\usepackage{amssymb}
%\usepackage{relsize}
%\usepackage[cmex10]{amsmath}
%\usepackage{amsthm}
%\interdisplaylinepenalty=2500
%\savesymbol{iint}
%\usepackage{txfonts}
%\restoresymbol{TXF}{iint}
%\usepackage{wasysym}
%\usepackage{amsthm}
%\usepackage{iithtlc}
%\usepackage{mathrsfs}
%\usepackage{txfonts}
%\usepackage{stfloats}
%\usepackage{bm}
%\usepackage{cite}
%\usepackage{cases}
%\usepackage{subfig}
%\usepackage{xtab}
%\usepackage{longtable}
%\usepackage{multirow}
%\usepackage{algorithm}
%\usepackage{algpseudocode}
%\usepackage{enumitem}
%\usepackage{mathtools}
%\usepackage{tikz}
%\usepackage{circuitikz}
%\usepackage{verbatim}
%\usepackage{tfrupee}
%\usepackage{stmaryrd}
%\usetkzobj{all}
%    \usepackage{color}                                            %%
%    \usepackage{array}                                            %%
%    \usepackage{longtable}                                        %%
%    \usepackage{calc}                                             %%
%    \usepackage{multirow}                                         %%
%    \usepackage{hhline}                                           %%
%    \usepackage{ifthen}                                           %%
  %optionally (for landscape tables embedded in another document): %%
%    \usepackage{lscape}
%\usepackage{multicol}
%\usepackage{chngcntr}
%\usepackage{enumerate}

%\usepackage{wasysym}
%\documentclass[conference]{IEEEtran}
%\IEEEoverridecommandlockouts
% The preceding line is only needed to identify funding in the first footnote. If that is unneeded, please comment it out.

\newtheorem{theorem}{Theorem}[section]
\newtheorem{problem}{Problem}
\newtheorem{proposition}{Proposition}[section]
\newtheorem{lemma}{Lemma}[section]
\newtheorem{corollary}[theorem]{Corollary}
\newtheorem{example}{Example}[section]
\newtheorem{definition}[problem]{Definition}
%\newtheorem{thm}{Theorem}[section]
%\newtheorem{defn}[thm]{Definition}
%\newtheorem{algorithm}{Algorithm}[section]
%\newtheorem{cor}{Corollary}
\newcommand{\BEQA}{\begin{eqnarray}}
\newcommand{\EEQA}{\end{eqnarray}}
%\newcommand{\define}{\stackrel{\triangle}{=}}
\theoremstyle{remark}
\newtheorem{rem}{Remark}

%\bibliographystyle{ieeetr}
\begin{document}
\title{LATEX ASSIGNMENT}
\author{VOORIBINDI CHANDINI}
\date{23-09-2023}
\maketitle
\section*{CLASS 10} 
\subsection*{Circles}
\date{}
\maketitle

\begin{enumerate}[label=\arabic*.,ref=\theenumi]
    \item In \figref{fig:fig1.png} if tangents $\vec{PA}$ and $\vec{PB}$ from an external point $P$ to a circle with centre $O$ , are inclined to each other at an angle of $80^{\degree}$, then $\angle AOB$ is equal to
 \begin{figure}[H]
        \centering
        \includegraphics[width = \columnwidth]{figs/figure1.png}
        \caption{Tangents $PA$ and $PB$}
        \label{fig:fig1.png}
    \end{figure}
    \begin{enumerate}
        \item $100^{\degree}$
        \item $60^{\degree}$
        \item $100^{\degree}$
        \item $100^{\degree}$
    \end{enumerate}

    \item  Two concentric circles are of radii $4 cm$ and $3 cm$. Find the length of the chord of the larger circle which touches the smaller circle.

    \item  In \figref{fig:fig2.png}, a triangle $ABC$ with $\angle AOB$ is shown. Taking $AB$ as diameter, a circle has been drawn intersecting $AC$ at point $P$. Prove that the tangent drawn at point $P$ bisects $BC$. 
	        \begin{figure}[H]
 %       \centering
        \includegraphics[width = \columnwidth]{figs/figure2.png}
        \caption{Concentric circles}
        \label{fig:fig2.png}
               \end{figure}
    \item  Prove that a Parallelogram circumscribing a circle is a rhombus.
     \item  In \figref{fig:fig3.png}, two circles with centres at $O$ and $O$ of radii $2r$ and $r$ respectively, touch each other internally at $A$. A chord $AB$ of the bigger circle meets the smaller circle at $C$. Show that  $C$ bisects $AB$.
    \begin{figure}[H]
        \centering
        \includegraphics[width = \columnwidth]{figs/figure3.png}
        \caption{Two circles with center}
        \label{fig:fig3.png}
    \end{figure}  
    
    \item In \figref{fig:fig4.png}, $O$ is centre of a circle of radius $5 cm$. $PA$ and $BC$ are tangents to the circle at $A$ and $B$ respectively. If $OP = 13 cm$, then find the length of tangents $PA$ and $BC$.
     \begin{figure}[H]
        \centering
        \includegraphics[width = \columnwidth]{figs/figure4.png}
        \caption{The center of the circle of radius 5 cm}
	     \label{fig:fig4.png}
    \end{figure}
    
    \item In two concentric circles, a chord of length $48 cm$ of the larger
circle is a tangent to the smaller circle, whose radius is $7 cm$. Find the radius of the larger circle. 
    \item At a point on the level ground, the angle of elevation of the top
of a vertical tower is found to be $\alpha$, such that $\tan \alpha =\frac{5}{12} $. On walking $192 m$ towards the tower, the angle of elevation $\beta$ is such that $\tan \beta=\frac{3}{4}$. Find the height of the tower. 
    \end{enumerate}
\end{document}
